\subsection*{Assignment overview}
\noindent You have two tasks in this assignment.

    \vspace{8pt}\noindent {\bf Task \#1}: You will write a small Python program to read a file containing clustered point data and plot it.

\vspace{4pt}\noindent {\bf Task \#2}: You will review your k-means code from the previous assignment and explain what changes must be made in order to ``parallelize'' it.

\vspace{8pt}\hrule 

\subsection*{Task \#1: Visualizing cluster data with Python}

\subsubsection*{Data file format}

The data file you will receive will have the same format as the output that you produced in homework 7.
\begin{itemize}
    \item \underline{Line 1}: A text header that should be read but ignored by your program.
    \item \underline{All other lines}: The data, where each line consists of three numbers separated by one space.
    The first two numbers of each line are geometric coordinates, and the third number is a cluster label.
    For example, the file may look like this:
    \begin{verbatim}
x              y        Cluster
-0.31611816 1.318305910 0
-0.13527147 -1.18088550 2
1.934038780 -0.66946771 3
1.866830940 -0.49043796 1
0.365777070 1.067407250 3
0.234074240 -0.49654951 1
-0.71434134 0.548718770 0
0.597211690 -0.04877915 1
0.672386510 -1.23401157 0
0.303697160 -0.37714570 2
    \end{verbatim}
\end{itemize}

\subsubsection*{Command line arguments in Python}

To see the parameters passed to your program from the command line,
you need to import the parameters from \texttt{sys} like this:
\begin{tsession}{mycodebg}
\begin{verbatim}
from sys import argv\end{verbatim}
\end{tsession}
This makes a list of strings called \texttt{argv}.
\texttt{argv[0]} will be the name of the file being executed (for you, this will be \texttt{"plotClusters.py"}).
For this assignment, \texttt{argv[1]} will be the name of the file you are to read.

\subsubsection*{Implementation}

\begin{tcolorbox}[width=\textwidth,colback=green]
To helps us find and test your code, your main Python file should be \texttt{HW08/plotClusters.py} (a file named \texttt{plotClusters.py} in a directory named \texttt{HW08}).
\end{tcolorbox}
You should write a Python 3 script that does the following:
\begin{enumerate}
    \item[Q1:] (5pts) Read a filename as input from the command line.
    For example the command
    \begin{tsession}{mytermbg}
    \begin{verbatim}
python3 plotClusters.py clusterdata.txt\end{verbatim}
    \end{tsession}
    \noindent should use a file named \texttt{clusterdata.txt} in the next step.
    \item[Q2:] (5pts) Read the data from the file.
    You may assume that the data is formatted as described earlier.
    \item[Q3:] (5pts) Store the data with NumPy array(s).
    You may want to store the data in Python list(s) at first and then convert them to array(s).
    \item[Q4:] (10pts) Plot the points of data as (x,y) pairs.
    Points should be plotted in the same color when they have the same cluster number.
    
    \underline{You may have a large number of clusters}.
    There are only eight colors that have names in pyplot (like \texttt{'b'} for blue).
    In order to give a different color to each cluster,
    you will need to generate custom colors in RGB format.
    For example, if you want to use a random color for each cluster,
    you could make a random array with the command
    \begin{tsession}{mycodebg}
    \begin{verbatim}
my_colors = np.random.rand(num_clusters,3)\end{verbatim}
    \end{tsession}
    This makes a \texttt{num\_colors}-by-3 array with random entries from 0 to 1.
   Each row of the array may be used as the \texttt{color} parameter of the \texttt{pyplot.plot} function.
\end{enumerate}

\subsubsection*{Testing}

Instructors will use Git to push \texttt{HW08/clusterdata.txt} to your repository on \href{http://code.vt.edu}{code.vt.edu}.
This file will have the format described earlier.
You will need to use \texttt{git pull} to bring the file in your local repository.

\begin{itemize}
    \item[Q5:] (5pts) Run your program to process the provided data file and save the plot as \texttt{HW08/cluster.pdf}
\end{itemize}

\vspace{8pt}\hrule

\newpage

\subsection*{Task \#2: Parallelizing k-means using OpenMP}

In a later assignment, you will be asked to modify your k-means code to make it run with multi-threading using OpenMP.
You do NOT need to modify your C code for this assignment.
\begin{tcolorbox}[width=\textwidth,colback=green]
If your code from Homework 7 did not work, you may use the instructor's code that is posted on Canvas \href{https://canvas.vt.edu/courses/95914/files/folder/Homework/HW07\%20example}{here}.
Read and understand this code and determine what changes will need to take place in order to parallelize it.
\end{tcolorbox}

\begin{itemize}
    \item[Q7:] (15pts) Analyze your k-means code and  explain how you plan to make it parallel using OpenMP. Refer to the checklist of properties that the code must satisfy to be eligible for OpenMP parallelism.
    
    Make a list of the changes that you will need to make in the sections of your code that:
    \begin{enumerate}
        \item [i.] Computes the centroids of the clusters.
        \item[ii.] Assigns the data points to the clusters.
    \end{enumerate}
    You should include the relevant pieces of code in your report (not the whole listing). Use full sentences to explain what you will need to do.
    
    You should explain in particular which loops may be made parallel using the OpenMP \texttt{\#pragma omp parallel for} directive. Note any sections of the body of these for loops that require special handling to avoid race conditions.
\end{itemize}
\vspace{8pt} \hrule
\newpage
\subsection*{Submission}

\begin{itemize}
    \item[Q8:] (5pts) Include the following in your \LaTeX{} typeset report:
\begin{itemize}
    \item Your Python code formatted using the \texttt{minted} environment.
    \item The plot that your code produced.
    \item An analysis of the parts of your k-means code that need to be adapted to make the code parallel using OpenMP.
\end{itemize}
\end{itemize}
Submit your assignment as follows:
\begin{itemize}
    \item Upload your PDF and tex files to Canvas.
    \item Submit the Python source file by pushing it to your git repository on \href{http://code.vt.edu}{code.vt.edu}.
    You may not receive any credit if your code is not in your repository.
\end{itemize}

