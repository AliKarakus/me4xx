
\subsection*{Pledged assignment}
This assignment is ``pledged.''
You can talk to anyone for help with \texttt{git} and \LaTeX{},
but for the coding portion you may ONLY get help from
the course instructors and the CMDA computing consultants (\href{https://www.ais.science.vt.edu/cmda/our-program/cmdacomputingconsultants.html}{information here}).

\vspace{8pt}\noindent At the top of your assignment, you will write
\begin{center}
``I have neither given nor received unauthorized assistance on this assignment.''
\end{center}

\subsection*{Instructor-provided code}
\begin{tcolorbox}[width=\textwidth,colback=green]
Completing this assignment requires that your code from the previous assignment works correctly.
If your code from HW 6 does not work well,
you may use the instructor's version instead.
This working version is posted on Canvas \href{https://canvas.vt.edu/courses/95914/files/folder/Homework/HW06\%20example}{here}.
There are two files:
\begin{itemize}
    \item \texttt{array.h} which defines a data struct and functions for getting and setting values, construction, a destruction, and finding the mean.
    \item \texttt{average.c} which defines a function for reading two-dimensional data from a file and a main function that ties everything together.
\end{itemize}
You may modify this code in order to complete this assignment.
\end{tcolorbox}

\subsection*{Assignment overview}
\vspace{8pt}\noindent In this assignment you will complete your K-means algorithm, test it for memory errors, and visualize its output.


\subsubsection*{Summary of workflow}

Here is an outline of how you should process data, start to finish.
Later sections describe details of each step as well as separate tasks like checking for memory errors.
\begin{enumerate}
    \item Your C program reads the data from the file provided.
    \item Your C program uses k-means to create clusters with the data.
    \item Your C program writes the cluster information to a file in a format that will compile in \LaTeX{}.
    \item Your tex document uses the cluster file as input to create a colored scatter plot.
\end{enumerate}

\subsubsection*{Data file format}

The data files you will receive will have the same format as in HW 6.
As a reminder, the files will have the following format:
\begin{itemize}
    \item \underline{Line 1}: A text header that should be read but ignored by your program.
    \item \underline{Line 2}: A number telling you how many lines of data there are.
    \item \underline{Line 3}:  A text header that should be read but ignored by your program.
    \item \underline{All other lines}: The data, where each line consists of two decimal numbers with one space between them.
    For example, the file may look like this:
    \begin{verbatim}
        <Number of data lines>
        3
        <Data>
        0.12345 6.78910
        3.14159 -2.0002
        1.41421 0.99999
    \end{verbatim}
\end{itemize}

\subsubsection*{\LaTeX{} Plotting}

You will use \LaTeX{} to produce a scatter plot for visualizing your data and assigned clusters.
An example of a \LaTeX{}-produced plot is found in \href{https://canvas.vt.edu/courses/95914/files/folder/Homework/LaTex\%20Plotting\%20Example#}{this directory on Canvas}.
You should look at these files:
\begin{itemize}
    \item \texttt{scatterdata.txt}: A data file with columns specifying the x and y coordinates of points and the cluster number that they belong to.
    Your C code will write a file like this for your own data using the clusters that you determine with k-means.
    \item \texttt{scatter.tex}: A small file showing how to make \LaTeX produce a scatter plot.
    Copy the \texttt{usepackage} statement into your preamble.
    Copy the \texttt{tikzpicture} portion into the tex file of your own final report.
    Modify the title, colors, and axis size to look good.
    \item \texttt{scatter.pdf}: The document produced by \texttt{scatter.tex} with the data from \texttt{scatterdata.txt}.
\end{itemize}

\subsubsection*{Implementation}

\begin{tcolorbox}[width=\textwidth,colback=green]
To helps us find and test your code, your main C file should be \texttt{HW07/kmeans.c} (a file named \texttt{kmeans.c} in a directory named \texttt{HW07}).
\end{tcolorbox}
You should write a C program that does the following:
\begin{enumerate}
    \item[Q1:] (5pts) Take a filename AND a number of clusters as input from the command line,
    read the data from the file,
    and store the information in a \texttt{struct} that you design.
    
    The data filename should be the first command line parameter followed by the number of clusters.
    For example, if your executable file is called \texttt{kmeans},
    the command
    \begin{tsession}{mytermbg}
    \begin{verbatim}
./kmeans test.dat 7\end{verbatim}
    \end{tsession}
    should read \texttt{test.dat} and make 7 clusters with the data.
    \item[Q2:] (30pts) Perform the K-means algorithm (Lloyd's algorithm variant) on the data to make the given number of clusters.
    \item[Q3:] (10pts) Write the cluster information to a file using the format of \texttt{scatterdata.txt} in the \LaTeX{} plotting example above.
    This file will be used as input in the tex code of your report.
\end{enumerate}

\subsubsection*{Coding style}

\begin{itemize}
    \item[Q4:] (10pts) Write your code with good style.
    A large discussion of C style is available \href{https://users.ece.cmu.edu/~eno/coding/CCodingStandard.html}{here}.
    The most important points that we will look for are
    \begin{itemize}
        \item \emph{Variable names}:
        give descriptive names to functions and important variables.
        Their names should relate to their purposes.
        \item \emph{Comments}:
        use comments to explain important sections of your code.
        \item \emph{White-space}:
        use matching indentation to emphasize code blocks (e.g. the bodies of loops and functions).
        Use line breaks to visually group closely related lines.
        \item \emph{Code block length}:
        code blocks should not be excessively long.
        If a function or loop is longer than about 20 statements,
        there is probably a good way to put some of that code into a sub-function with a descriptive name.
        \item \emph{Flexible code}:
        Your code should be able to handle input that is different from what we have asked for.
        In particular, your code should use the heap in order to be able to handle large files.
        It should also be able to run on (properly formatted) files of different sizes (so do not ``hard code'' a number of lines for the file).
    \end{itemize}
\end{itemize}

\subsubsection*{Testing}

Instructors will use Git to push two files into a directory called \texttt{HW07} in your repository on \href{http://code.vt.edu}{code.vt.edu}.
These files are
\begin{itemize}
    \item  \texttt{test.dat}: a data file with the format shown above.
    \item \texttt{instructions.txt}: a text file telling you how many clusters you should make.
    Open this file and read it with your eyes.
    Use the stated number of clusters as an input for your program.
\end{itemize}
You will need to use \texttt{git pull} to bring the files in your local repository.

\begin{itemize}
    \item[Q5:] (5pts) Run your program.
    Use the command line arguments \texttt{test.dat} and the number of clusters specified in \texttt{instructions.txt}.
    The output file will be used for your report.
\end{itemize}

\subsubsection*{Memory errors}

When using pointers and \texttt{malloc},
programmers need to be careful not to misuse memory
or fail to \texttt{free} memory that they allocated.

Valgrind is a tool for checking programs for memory errors.
You can install it with \texttt{apt-get} as usual:
\begin{tsession}{mytermbg}
\begin{verbatim}
sudo apt-get valgrind
\end{verbatim}
\end{tsession}
To use Valgrind, simply put \texttt{valgrind} in front of your command to run your program.
If your executable is called \texttt{go} and you want to make 7 clusters out of the data in \texttt{test.dat},
you can check for memory errors with the command
\begin{tsession}{mytermbg}
\begin{verbatim}
valgrind ./kmeans test.dat 7
\end{verbatim}
\end{tsession}

\begin{itemize}
    \item[Q6:] (10pts) Run your program again with Valgrind.
    No leaks or errors should be reported.
\end{itemize}

\subsubsection*{Submission}

\begin{itemize}
    \item[Q7:] (10pts) Include the following in your \LaTeX{} typeset report:
\begin{itemize}
    \item Your C code formatted using the \texttt{minted} environment.
    \item A color-coded plot of your clusters.
    Use the plotting features of \LaTeX{} explained in class to plot the data and the centers of the clusters you made.
    Each point that is in the same cluster should be plotted in the same color.
    Points in different clusters should be plotted in different colors.
    \item The output of Valgrind checking your code.
\end{itemize}
\end{itemize}
Submit your assignment as follows:
\begin{itemize}
    \item Upload your PDF and tex files to Canvas.
    Be sure to include the data cluster file that your C program produces for plotting.
    \item Submit the C source file by pushing it to your git repository on \href{http://code.vt.edu}{code.vt.edu}.
    You may not receive any credit if your code is not in your repository.
\end{itemize}