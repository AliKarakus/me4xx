
\noindent \textbf{Directions}: 

\begin{enumerate}
\item Use \LaTeX{} to create at least one of the two options. 
\item A score of 20 points will be considered full credit.

\item Save the .tex file as 
\begin{center}
[LAST NAME]\_[FIRST NAME]\_HW04\_Option[NUMBER].tex
\end{center} 
\item When finished, upload BOTH the tex file and the pdf file created to Canvas by clicking on the Assignments tab on the left side of the CMDA3634 Canvas page.
\item You may choose to attempt both options for a maximum of up to 30 points. If you attempt both options upload both sets of files.
\end{enumerate}

\newpage
\section*{Option 1 [20 points]}
Starting with a template from
\begin{center}
\url{https://www.sharelatex.com/templates/cv-or-resume}
% https://www.overleaf.com/gallery/tagged/cv
\end{center}
produce your own CV (curriculum vitae) describing your activities relevant to seeking a job.
The document does not need to be limited to a single page,
but it should not be significantly less than one page.
For this class, you won't be graded on the content of your CV.
If you need to fill out more space, you may invent fictional activities (e.g. 2018-present: elephant wrangler, and that is in fact a real job involving a large shovel).

Your CV should be professional looking and a document that you can keep updating as you gain more experience.

\subsection*{Rubric}
\begin{itemize}
    \item (10 points) Successful communication of activities using sections and spacing in a way that makes navigation of the document easy.
    \item (10 points) Overall professional appearance.
\end{itemize}
\begin{center}\rule{0.8\textwidth}{.4pt}\end{center}

\newpage
\section*{Option 2 [20 points]}
After the table on the next page, there are two pages with mathematical text.
Reproduce this text as closely as possible, but replace ``Your Name'' with your actual name.
Pay attention to indentation, alignment, brackets, punctuation, symbols, section headers, and spacing.
You should find yourself using the \LaTeX{} features in the table on the next page.

\subsection*{Rubric}
\begin{itemize}
    \item (10 points) Content: having the correct words, variables, and symbols.
    \item (10 points) Style: using the correct spacing and formatting.
\end{itemize}

\begin{center}\rule{0.8\textwidth}{.4pt}\end{center}

\begin{center}
\begin{tabular}{l|l} \hline
Feature & Purpose \\ \hline \hline
\texttt{\tbs documentclass[11pt]\{article\}} & declare the type of document and \\&set the default text size\\ \hline
\texttt{\tbs usepackage\{amsfonts\}} & load the blackboard bold font (e.g. $\mathbb{R}$)\\
\texttt{\tbs usepackage\{amsthm\}} & load theorem and proof environments\\
\texttt{\tbs newtheorem\{thm\}\{Theorem\}} & define an environment called ``\texttt{thm}''\\
& for making theorem statements. The word\\
&``\texttt{Theorem}'' will be inserted at the beginning\\
&of this environment\\
\texttt{\tbs author\{Your Name\}}  & preamble macro to define author's name \\ 
\texttt{\tbs title\{Document Title\}}  & preamble macro to define title of paper \\ \hline
\texttt{\tbs begin\{document\}}  & start of body of document \\
\texttt{\tbs end\{document\}}  & end of body of document \\ \hline
\texttt{\tbs maketitle}  & macro to create title \\
\texttt{\tbs begin\{abstract\}}  & start of abstract text \\
\texttt{\tbs end\{abstract\}}  & end of abstract \\ \hline
\texttt{\tbs begin\{thm\}}  & start theorem statement \\
\texttt{\tbs end\{thm\}}  & end the theorem statement \\ \hline
\texttt{\tbs begin\{proof\}}  & start theorem proof \\
\texttt{\tbs end\{proof\}}  & end the theorem proof \\ \hline
\texttt{\tbs section\{Section Title\}}  & insert section header. \\ \hline
\texttt{\tbs[}  & start local math environment,\\&centered and on a separate line \\ 
\texttt{\tbs]}  & end local  math environment \\
\texttt{\$ math stuff \$} & math environment within the paragraph\\
\hline
\texttt{\tbs sqrt\{argument\}}  & display square root of argument \\
\texttt{x\string^2}  & raise the baseline of 2 to the power position \\ 
\texttt{\tbs frac\{a\}\{b\}}  & fraction macro for $\frac{a}{b}$\\
\texttt{\tbs left(}  & left parenthesis that grows to match contents \\
\texttt{\tbs right)}  & matching right-parenthesis \\
\texttt{\tbs in}  & in symbol ($\in$) \\
\texttt{\tbs neq}  & not equal sign ($\neq$) \\
\texttt{\tbs pm}  & plus-minus symbol ($\pm$) \\
\hline

\end{tabular}
\end{center}

\newpage
%\clearpage
%\setcounter{page}{1}

%\newpage
% \maketitle
 
% \begin{abstract}
%% hacked for the moment
\begin{center}
{\bf Abstract}: \emph{A brief proof of the classical formula for the roots of a quadratic polynomial.}
\end{center}
% \end{abstract}
 
\section{Introduction}
In this section we present a brief proof of the classical formula used to obtain the roots of a quadratic polynomial. 

\begin{thm}
Given $a,b,c\in \mathbb{R}$ with $a\neq 0$ the following equation 
\[
ax^2 + bx + c = 0,
\]
has at most two solutions given by
\[
x = \frac{ -b \pm \sqrt{b^2 - 4ac}}{2a}.
\]
\end{thm}
\begin{proof}
Given
\[
ax^2 + bx + c = 0,
\]
we first multiply through by $a$
\[
a^2x^2 + abx + ac = 0,
\]
and  we subsequently ``complete the square''
\[
\left(ax + \frac{b}{2}\right)^2 - \frac{b^2}{4} + ac = 0.
\]
Next we move terms to the right hand side
\[
\left(ax + \frac{b}{2}\right)^2 = \frac{b^2}{4} - ac.
\]
Taking the square root of either side
\[
ax + \frac{b}{2} = \pm \sqrt{\frac{b^2}{4} - ac},
\]
and rearranging gives
\[
ax = -\frac{b}{2} \pm \sqrt{\frac{b^2}{4} - ac}.
\]
Finally since by assumption $a\neq 0$ we divide both sides by $a$, factor $1/4$ out of the square root, and express the whole thing as a convenient single fraction 
\[
x = \frac{ -b \pm \sqrt{b^2 - 4ac}}{2a},
\]
concluding the proof.
\end{proof}
\section{Conclusion}
In this article we have presented a proof of the classical formula for roots of a quadratic polynomial.