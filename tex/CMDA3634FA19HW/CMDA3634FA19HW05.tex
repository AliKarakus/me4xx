
\subsection*{Pledged assignment}
This assignment is ``pledged.''
You can talk to anyone about \texttt{gcc} and \LaTeX{},
but you may not get help from other students on the coding portion.
You can talk to the course instructors and the CMDA computing consultants (\href{https://www.ais.science.vt.edu/cmda/our-program/cmdacomputingconsultants.html}{information here}).

\vspace{8pt}\noindent At the top of your assignment, you will write
\begin{center}
``I have neither given nor received unauthorized assistance on this assignment.''
\end{center}

%NO CODE in assignment statement

\subsection*{Instructions}
In this assignment you will investigate a basic difference between the real numbers ($\mathbb R$) and floating-point numbers by considering the sequence of numbers described by
    \begin{align}
        a_0&=\frac13,\\
        a_{n+1}&=4a_n-1. \label{recur.eqn}
    \end{align}
    
\begin{enumerate}
    \item[Q1:] (5pts) What is $a_{50}$ in exact arithmetic? Can you write down what $a_n$ is for all $n$ if the recurrence is evaluated with exact arithmetic?
    \item[Q2:] (10pts) Now consider the sequence of numbers $\{a_n\}$ that are obtained when using finite precision floating point arithmetic.
    
    Write a C program that uses single precision \texttt{float} variables to evaluate the Equation \ref{recur.eqn} to obtain $a_0,\dots,a_{50}$ and print this sequence of numbers to the terminal to 15 significant digits.
   
    \item[Q3:] (5pts) Create a new implementation of Q2 that uses double precision \texttt{double} type variables. 
    \item[Q4:] (5pts) Explain why $a_{50}$ found in Q2 and Q3 is so different from your answer in Q1.
    Use what you know about how computers store floating point variables.
\end{enumerate}

\subsection*{Submission}
\begin{itemize}
    \item[Q5:] (5pts) Include the following in your \LaTeX{} typeset report:
    \begin{enumerate}
        \item Your C code formatted using the \texttt{minted} environment.
        \item Your responses to the above questions. 
        \item The output of your computer programs, nicely formatted.
    \end{enumerate}
 
    Submit your assignment as follows:
    \begin{itemize}
    \item Upload your PDF and tex files to Canvas.
    \item Submit the C source files from Q2 an Q3 by pushing them to your git repository on \href{code.vt.edu}{code.vt.edu}.
    \end{itemize}
\end{itemize}
