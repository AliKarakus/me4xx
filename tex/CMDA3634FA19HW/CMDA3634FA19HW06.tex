In this assignment you will write C code that will support your k-means algorithm.
\subsection*{Grant permission}
The GitLab settings we originally asked you to use will not allow us to push the files you need.
If you have not done so already, please do the following as soon as possible:
\begin{enumerate}
    \item Go to \href{http://code.vt.edu}{code.vt.edu}.
    \item Select your cmda3634 project.
    \item In the left column, navigate to ``Settings'' then ``Members.''
    \item At the bottom of the page, use the drop-down menus to change the permissions of \texttt{jonpbak} and \texttt{tcew} to ``Maintainer.''
\end{enumerate}
\subsection*{Implementation}
\begin{tcolorbox}[width=\textwidth,colback=green]
To helps us find and test your code, your C file should be \texttt{HW06/average.c} (a file named \texttt{average.c} in a directory named \texttt{HW06}).
\end{tcolorbox}
Implement the following as a C program:
\begin{enumerate}
    \item[Q1:] (5pts) Accepts a command line argument to specify the name of a file to open for reading.
    
    For instance if your executable file is named \texttt{go}, then the command
    \begin{tsession}{mytermbg}
    \begin{verbatim}
./go bah.dat\end{verbatim}
    \end{tsession}
    should open the file named \texttt{bah.dat} to read.
   
    \item[Q2:] (10pts) Reads numerical two-dimensional data from the file.
    The file will be formatted as follows:
    \begin{itemize}
        \item \underline{Line 1}: A text header that should be read but ignored by your program.
        \item \underline{Line 2}: A number telling you how many lines of data there are.
        \item \underline{Line 3}: A text header that should be read but ignored by your program.
        \item \underline{All other lines}: The data, where each line consists of two decimal numbers with one space between them.
    \end{itemize}
    For example, the file may look like this:
    \begin{verbatim}
        <Number of data lines>
        3
        <Data>
        0.12345 6.78910
        3.14159 -2.0002
        1.41421 0.99999
    \end{verbatim}
    \item[Q3:] (5pts) Stores the data from the file into a data structure specified by a \texttt{struct} of your design. 
    \item[Q4:] (5pts) Finds the mean (average) of the data (column-wise).
    For the example file above, the mean is
    \begin{verbatim}
        1.55975 1.92963
    \end{verbatim}
    \item[Q5:] (5pts) Displays the mean with 5 significant digits.
\end{enumerate}

\subsection*{Testing}
Instructors will use Git to push several files into a directory called \texttt{HW06} in your repository on \href{http://code.vt.edu}{code.vt.edu}.
You will need to use \texttt{git pull} to put these files in your local repository.

The files are as follows

% https://code.vt.edu/tcew/cmda3634/blob/master/L09/arrayA.dat

\begin{itemize}
    \item \texttt{anwers.txt}: The correct means of \texttt{easy.dat} and \texttt{debug.dat}.
    \item \texttt{easy.dat}: A data set whose average you should be able to find in your head.
    \item \texttt{debug.dat}: A large data set for debugging your algorithm.
    You will be given the correct mean of this data (in \texttt{answers.txt}) so that you can make sure your algorithm works.
    \item \texttt{test.dat}: A large for grading your algorithm.
    You will not be given the mean for this data set.
    \item[Q6:] (10pts) Run your program with each of the \texttt{.dat} files as input.
    This means that you will run your program three times and record the results and include them in your report.  Note: you are not allowed to change the source file between different runs.
\end{itemize}

\subsection*{Submission}
\begin{itemize}
    \item[Q7:] (5pts) Include the following in your \LaTeX{} typeset report:
    \begin{enumerate}
        \item Your C code formatted using the \texttt{minted} environment.
        \item The text output of your computer program using a \texttt{verbatim} environment (not a screenshot). Include the results of the runs for all three data files.
    \end{enumerate}
 
    Submit your assignment as follows:
    \begin{itemize}
    \item Upload your PDF and tex files to Canvas.
    \item Submit the C source file by pushing them to your git repository on \href{http://code.vt.edu}{code.vt.edu}.
    \end{itemize}
\end{itemize}
