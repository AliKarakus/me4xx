\subsection*{Setting up}
Create a Git repository%
\footnote{On GitLab---which is used by code.vt.edu---a ``project'' is a repository (collection of code with version tracking) along with associated comments, bug tracking tools, etc.}
and share it with your instructors.
A brief outline of the steps is below.
Detailed instructions are in Chapter 3 of the notes.
\begin{enumerate}
\item Create an account and git repository on code.vt.edu.
\begin{itemize}
    \item Make your project private.
    \item Name your project \texttt{cmda3634}.
\end{itemize}
\item Grant developer access to \texttt{tcew} and \texttt{jonpbak}.
\item Install Git on your VirtualBox.
\item Create an ssh key and add it to your repository.
\item Use the command line to clone your repository to your VirtualBox (a new directory will be created).
\end{enumerate}
\subsection*{What to submit}
On Canvas, submit your PID and the URL that you used to clone your repository (starts with \texttt{git@}).
This can be the text portion of your submission.

Submit a screenshot showing a terminal printout of a successful repository setup.
\begin{enumerate}
    \item Use \texttt{pwd} to confirm that you are in the \texttt{cmda3634} directory.
    \item Use \texttt{git status} to confirm that the directory is a Git repository.
    \item Use \texttt{gnome-screenshot} to save a screenshot displaying the results of the last two commands.
    \item Upload this screenshot to Canvas as your assignment.
\end{enumerate}