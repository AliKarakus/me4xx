\subsubsection*{Assignment overview}
\noindent You have two tasks in this assignment.

\vspace{8pt}\noindent {\bf Task \#1}: Modify the code in \texttt{parallelQuiz.c} to make as many of its loops run in parallel as possible. In your report, explain which loop(s) cannot be parallelized and why some of them work without modification.
\\
\vspace{4pt}\noindent {\bf Task \#2}: Use OpenMP to make your k-means code run in parallel.
\begin{tcolorbox}[width=\textwidth,colback=green]
If your code from Homework 7 did not work, you may use the instructor's code that is posted on Canvas \href{https://canvas.vt.edu/courses/95914/files/folder/Homework/HW07\%20example}{here}.
Read and understand this code and determine what changes will need to take place in order to parallelize it.
\end{tcolorbox}
\vspace{8pt}\hrule 

\subsubsection*{Task \#1: Parallelization quiz}
You will parallelize some for-loops.
You will have a section in your report to answer questions about the loops,
and you will implement any necessary changes in the C file itself.
\begin{itemize}
    \item[Q1] (25pts) For each for-loop, below (A through K), read the comment describing its intended behavior.
    Then decide which of the following statements applies.
    Explain your answer as indicated.
    \begin{itemize}
        \item \underline{The loop can be parallelized without modification}.
        Use the loop analysis methods shown in class to justify your answer.
        
        In the C file itself, insert the OMP compiler directive to make the loop run in parallel.
        \item \underline{The loop can be parallelized, but the code inside must be modified}.
        In this case, write a parallelized version of the loop (formated with the \texttt{minted} environment).
        Your version must follow any instructions in the comment above the loop.
        Unless the loop-comment says otherwise, you may use any method you like to produce the desired behavior.
        Your version must produce exactly the same result.
        
        Also implement your changes in the C file itself, including adding the OMP parallelization compiler directive.
        \item \underline{The loop CANNOT be parallelized}.
        In this case, explain WHY the task cannot be parallelized.
        
        Don't modify the loop in the C file.
    \end{itemize}
\end{itemize}
You are welcome to run tests to make sure any modifications that you make work correctly.
Rather than copying and pasting, you can download \texttt{parallelQuiz.c} \href{https://canvas.vt.edu/files/11298897/download?download_frd=1}{here}.
\newpage
\begin{minted}{c}
// parallelQuiz.c

#include <omp.h>
#include <stdlib.h>
#include <stdio.h>
#include <math.h>

// Declare functions that will be defined later
int collatzIter(int n);
int collatzTime(int n);
int alternateSigns();

int main(int argc, char **argv){

  int n, i, j, k;
  int N = 100;

  int *v = (int*) malloc(N*sizeof(int));
  int *w = (int*) malloc(N*sizeof(int));
  float *z = (float*) malloc(N*sizeof(float));

  // Globally set number of threads
  omp_set_num_threads(4);
  
  // Loop A
  // Fill v with zeros
  for(n=0;n<N;++n){
    v[n] = 0;
  }

  // Loop B
  // Starting from the end, fill array with values 0,1,2,..,N-1
  for(n=N-1;n>=0;--n){
    v[n] = n;
  }

  // Loop C
  // Fill v with values 0,3,6,..,3*N-3
  v[0] = 0;
  for(n=1;n<N;++n){
    v[n] = v[n-1]+3;
  }

  // Loop D
  // Reverse the entries of v in a single loop
  for(n=0;n<N/2;++n){ 
    i = N-1-n;
    j = v[n];
    v[n] = v[i];
    v[i] = j;
  }


  // Loop E
  // Propagate value from v[0] to all entries of v
  for(n=1;n<N;++n){
    v[n] = v[0];
  }

  // Loop F
  // Use ONLY alternateSigns() function to fill v with values 1,-1,1,..,-1
  for(n=0;n<N;++n){
    v[n] = alternateSigns();
  }

  // Loop G
  // Make v[n] the second term in n's Collatz sequence
  for(n=1;n<N;++n){
    v[n] = collatzIter(n);
  }

  // Loop H
  // Make v[n] the nth term of the Collatz sequence starting with 137
  v[0] = 137;
  for(n=1;n<N;++n){
    v[n] = collatzIter(v[n-1]);
  }

  // Loop I
  // Make v[n] the Collatz convergence time of n
  for(n=0;n<N;++n){
    v[n] = collatzTime(n);
  }

  // Loop J
  // Store v in reverse order in w
  for(n=0;n<N;++n){
    i = N-1-n;
    w[n] = v[i];
  }

  // Loop K
  // Record sine of v in z
  for(n=0;n<N;++n){
    z[n] = sin(v[n]);
  }
  free(v);
  free(w);
  free(z);
}

/*
DO NOT MODIFY THESE FUNCTIONS
You should think of them as library functions
that are outside of your control.
*/

int collatzIter(int n){
  // Perform one Collatz iteration
  if(n%2 == 0)
    return n/2;
  return 3*n + 1;
}

int collatzTime(int n){
  // Number of iterations to complete Collatz iteration
  int counter = 0;
  while(n > 1){
    n = collatzIter(n);
    ++counter;
  }
  return counter;
}

int sign = -1;
int alternateSigns(){
    // Returns 1, then -1, then 1, etc.
    sign *= -1;
    return sign;
}
\end{minted}

\begin{tcolorbox}[width=\textwidth,colback=pink]
You may ONLY modify the for-loops.
Any modifications should follow the directions in the comment above the loop.
\end{tcolorbox}

\newpage

\subsubsection*{Task \#2: Parallelize k-means}
Make a copy of your k-means code from HW 7 into your \texttt{HW09} directory.
Modify your project in the following ways:
\begin{itemize}
    \item[Q2:] (10pts) Parallelize the task of assigning data points to clusters using OpenMP directives.
    \item[Q3:] (10pts) Parallelize the task of computing cluster centroids  using OpenMP directives.
    \item[Q4:] (5pts) Use good coding style as explained in the lecture notes and the previous homework instructions.
    \item[Q5:] (5pts) Use separate directories to sort the several types of files involved in your project, such as \texttt{src} and \texttt{data}.
    \item[Q6:] (5pts) Provide a makefile called \texttt{makefile} that compiles your program simply by running \texttt{make} from your project root directory, \texttt{HW09}.
    \item[Q7:] (5pts) Include a screenshot of the terminal \texttt{top} command showing your OpenMP code running and attaining more than 100\% of CPU core utilization.
\end{itemize}

\subsection*{Submission}

\begin{itemize}
    \item[Q7:] (5pts) Submit your work as follows.
    \begin{itemize}
        \item Write a report on your results and use \LaTeX to typeset it. Your report should include
        \begin{itemize}
            \item A section analyzing each of the for-loops in \texttt{parallelQuiz.c}.
            For each of the loops that you believe must be modified, include the C code of your parellelized version formatted nicely using the \texttt{minted} environment.
            \item A separate subsection containing your C code from task \#2 formatted nicely using the \texttt{minted} environment.
        \end{itemize}
        \item Upload the PDF and tex files of your report to Canvas.
        \item Submit the C source files of BOTH tasks by pushing them to your Git repository on \href{http://code.vt.edu}{code.vt.edu}
        You may not receive any credit if your code is not in your repository.
        \begin{itemize}
            \item For task \#1, you should push your modified version of \texttt{parallelQuiz.c}.
            If your report says that a loop can be parallelized, it should have the OMP compiler directive above it
            \begin{tsession}{mycodebg}
            \begin{verbatim}
#pragma omp parallel for\end{verbatim}
            \end{tsession}
        \end{itemize}
    \end{itemize}
\end{itemize}
