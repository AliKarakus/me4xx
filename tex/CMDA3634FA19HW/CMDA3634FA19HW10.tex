
\subsection*{Pledged assignment}

This assignment is ``pledged.''
You can talk to anyone about tools of the class (gcc, Git, MPI, etc.), but only to help you understand concepts.
You can study generic examples of MPI code to understand how MPI works,
but you are not allowed to share or look for MPI code that solves the particular problems of this assignment.
\textcolor{red}{All of the code you submit must be your own work or taken from example code provided by instructors.}
In particular, you are not to copy code from other students or the Internet.
You can get extra help from the course instructors and the CMDA computing consultants (\href{https://www.ais.science.vt.edu/cmda/our-program/cmdacomputingconsultants.html}{information here}).

\vspace{8pt}\noindent At the top of your assignment, you will write
\begin{center}
``I have neither given nor received unauthorized assistance on this assignment.''
\end{center}

\subsection*{Assignment overview}

\noindent You have two tasks in this assignment.

\vspace{8pt}\noindent {\bf Task \#1}: Fix and improve the code in \texttt{mpiDebug.c}.
\\
\vspace{4pt}\noindent {\bf Task \#2}: Use code parallelized with MPI to find the centroids of the clusters described in a file.
\vspace{8pt}\hrule 

\subsection*{Task \#1: Parallel Debugging}

The code in \texttt{mpiDebug.c} (\href{https://canvas.vt.edu/files/11482160/download?download_frd=1}{download here}) attempts to read \texttt{mpiClusters.dat} and find the maximum and minimum x-coordinates in parallel using MPI.
\begin{itemize}
     \item[Q1.1] (10pts) Debug the code and include a corrected version in your report and Git repository.
    \item[Q1.2] (5pts) Make the code more efficient by letting rank 0 begin its calculation of the minimum without waiting until rank 1 has received the data.
\end{itemize}
\begin{tcolorbox}[width=\textwidth,colback=green]
Task \#1 is a parallel debugging exercise.
The code in \texttt{mpiDebug.c} has nothing to do with k-means or clustering.
The code reads from the cluster data file \texttt{mpiClusters.dat} for convenience only.
Don't try to change the purpose of \texttt{mpiDebug.c}.
It is for finding a maximum and a minimum, not for clustering or finding centroids.
\end{tcolorbox}
\subsection*{Task \#2: Parallel centroid calculation}
\subsubsection*{Test file format}
This assignment uses a data file format like those that have been used in earlier assignments.
The format of your input files will be as follows
\begin{itemize}
    \item Line 1: A header that your program should ignore.
    \item Line 2: A number telling you the total number of lines of data in the file.
    \item Line 3: A header that your program should ignore.
    \item All other lines: Space-separated numbers giving, in order, the x and y coordinates of data points (floating point values) and a cluster numbers (integers).
\end{itemize}
For example, the file contents might be
\begin{tsession}{mycodebg}
\begin{verbatim}
<Number of lines of data>
12
<Data>
0.978357560 -0.78691991 0
-0.38803385 -0.09377587 2
-0.60679914 0.158120691 1
-0.27996106 0.844734722 1
0.181741410 1.176679913 0
-0.18679137 0.474455854 0
1.550371030 0.219166765 0
-0.13938078 1.562948096 0
-1.02885595 0.051917867 2
-0.69919482 0.650696358 2
-0.63400276 -0.15668462 1
1.774652670 -0.26363615 1
\end{verbatim}
\end{tsession}
\subsubsection*{Your task}
Write a program in C that does the following
\begin{itemize}
    \item[Q2.1] (5pts) Takes a filename as a command line argument and reads the file's contents you may assume the file has the format above.
    Your program should use the values in the third column to determine the number of clusters.
    
    You should be smart about reading the file. Every process should not have to read all of the data and each process should only own (i.e. store) approximately $N/P$ data points where $N$ is the total number of data points and $P$ is the number of processes.
    
    \item[Q2.2] (20pts) For each cluster, compute the local contribution to the centroid calculation from the datapoints owned by the process.
    Then use \texttt{MPI\_Allreduce} to finalize the addition of the cluster centroids and to determine the number of datapoints in each cluster.
    
    Finally, all the processes use the information distributed by calls to \texttt{MPI\_Allreduce} to complete the calculation of the cluster centroids.
    All processes should do about the same amount of work.
    
    \item[Q2.3] (5pts) Nominate a single process to print the space-separated centroid data to standard output in the order of the cluster numbers ($0,1,\dots$).
    With the example file above, your program should print the following
    \begin{tsession}{mycodebg}
    \begin{verbatim}
0.47685957 0.52926614
0.06347243 0.14563366
-0.7053615 0.20294611\end{verbatim}
    \end{tsession}
\end{itemize}
You may download the large test file \texttt{mpiClusters.dat} \href{https://canvas.vt.edu/files/11482183/download?download_frd=1}{here}.
I do not need 50 copies of this file, so \textcolor{red}{DO NOT} \texttt{push} this data file to your Git repository.
We will run tests using our own data files.
\subsubsection*{Testing}
\begin{itemize}
    \item[Q2.4] (5pts) Run your program with \texttt{mpiClusters.dat} as input.
    Copy the printed result into your report in an equal-width environment like \texttt{verbatim}.
\end{itemize}

\subsection*{Submission}

\begin{itemize}
    \item[Q3.1] (5pts) Submit your work as follows.
    \begin{itemize}
        \item Write a report on your results and use \LaTeX{} to typeset it.
        \begin{itemize}
            \item  Your report should include
            \begin{itemize}
                \item Your improved version of \texttt{mpiDebug.c} from task \#1, formatted nicely using the \texttt{minted} environment.
                \item Your C code from task \#2, formatted nicely using the \texttt{minted} environment.
                \item The centroids of the clusters in \texttt{mpiClusters.dat}.
                \item A screenshot of a \texttt{top} readout displaying more than 100\% CPU usage for your program.
            \end{itemize}
            \item Upload the PDF and tex files of your report to Canvas.
        \end{itemize}
        \item Submit the C source files of BOTH tasks by pushing them to your Git repository on \href{http://code.vt.edu}{code.vt.edu}
        You may not receive any credit if your code is not in your repository.
    \end{itemize}
\end{itemize}