\chapterauthor{Tim Warburton}

\epigraph{A supercomputer is a device for turning compute-bound problems into I/O-bound problems}{Ken Batcher}

\minitoc

The Linux file system really is just a hierarchical collection of files. In the following we will discuss how to interact with the file system. Specifically we will start by using the C standard IO (input/output) library to open a  file, read the contents, then close the file. We will repeat this process but writing write data to a file. 

\section{Reading data from a file}

The C standard IO library internally represents a file using a struct of type \texttt{FILE}. We do not need to be concerned about the contents of a \texttt{FILE} struct, but conceptually we should be aware that once a file is open the library keeps track of our place in the file. 

The following code includes the \texttt{stdio.h} header file, opens a file, and reads an integer value from the file.

\begin{minted}{c}
/* L09/readIntFromFile.c */

#include <stdio.h>

int main(int argc, char **argv){
  
   FILE *fp = fopen("foo.dat", "r");
   
   int n;
   fscanf(fp, "%d", &n);
   
   printf("Read %d from foo.dat\n", n);
   
   fclose(fp);
   return 0;
}
\end{minted}

We first open the file using the file open function \texttt{fopen}. 
\begin{itemize}
\item[]{\bf Note \#1}: in this case we pass the name of the file as a constant string literal. 

\item[]{\bf Note \#2}: if the file is not in the present working directory we would have to include the path to the file in the file name string.

\item[]{\bf Note \#3}: we also pass a second string that specifies that we wish to \underline{\bf r}ead from the file. If we had wanted to write to the file we would have replaced the r in the string with w.

\item[]{\bf Note \#4}: if the file does not exist or cannot be opened for some reason \texttt{fopen} will return a NULL pointer. We should add code to check for this.
\end{itemize}

Once the file is opened we can proceed to scan through the file. We use the \texttt{fscanf} file-scan-formatted function in this case to read a single an integer. It will ignore white space (like tabs or blank spaces) and then try to read an integer. 

\begin{itemize}
\item[] {\bf Note \#1}: we instruct \texttt{fscanf} what kind of text we want to read and what kind of number to output through the format string. In this case \texttt{\%d} specifies that we want \texttt{fscanf} to treat the next group of non-space characters as a number. 

\item[]{\bf Note \#2}: once the number is read the standard IO library will advance its internal file pointer to the next character after the end of the number it read.

\item[]{\bf Note \#3}: you can specify that more than one object should be read by adding extra format blocks to the format string. Some options are: \texttt{\%c} for a \texttt{char}, \texttt{\%s} for a string, \texttt{\%f} for a float, etc. See \href{https://www.tutorialspoint.com/c_standard_library/c_function_fscanf.htm}{this tutorial} for more info.

\item[]{\bf Note \#4}: notice how we passed a pointer to the \texttt{n} variable. Since \texttt{fscanf} can read more than one variable from the file it is necessary to provide pointers to the variables passed in to receive the data from the file.
\end{itemize}

Finally we use \texttt{fclose} to close the file. It is always important to close files after we are done with them, in particular when writing data to them as in the next section.

To demonstrate scanning for more than one thing from a file see the following code

\begin{minted}{c}
/* L09/readMixedFromFile.c */
#include <stdio.h>

int main(int argc, char **argv){
  
   FILE *fp = fopen("foo.dat", "r");
   
   int n,m;
   float f;
   fscanf(fp, "%d %d %f", &n, &m, &f);
   
   printf("Read %d,%d,%f from foo.dat\n", n,m,f);
   
   fclose(fp);
   return 0;
}
\end{minted}

\section{Writing data to a file}

The process of writing data to a file is much as before. In the simplest approach we simply open a new file specifying that we will write to the file and then we use the file-print-formatted standard IO function \texttt{fprintf} to write data to the file.

\begin{minted}{c}
/* L09/writeMixedToFile.c */
#include <stdio.h>

int main(int argc, char **argv){
  
   /* open bah.dat in the pwd to write to */
   FILE *fp = fopen("bah.dat", "w");
   
   int n = 10,m = 20;
   float f = 1.3;

   fprintf(fp, "%d %d %f", n, m, f);
   
   fclose(fp);
   return 0;
}
\end{minted}
Note the action string passed to \texttt{fopen} is ``w'' for \underline{\bf w}rite. Also note subtle difference between \texttt{fprintf} and \texttt{fscanf}. Because we are passing data into \texttt{fprintf} we pass by value and do not need to pass pointers into the data arguments.

We can further tweak the format string passed to \texttt{fprintf} to specify the number of decimal places or whether we should scientific notation for the \texttt{float}. You can find formatting string information with your favorite search engine (search: \emph{C format string fprintf}) or you can use the builtin Terminal man pages with 
\myvbox[mytermbg]{man fprintf}
There are \texttt{man} pages for most C standard library functions accessible through the \texttt{man} command.

\section{Example: reading/writing arrays from/to text files}

In this section we will expand on the functionality of our \texttt{array} struct from the defensive programming  Section \ref{array.sec}.

We will specify that an \texttt{array} file has an expected format:
\myvbox[myoutputbg]{<NUMBER OF ENTRIES IN ARRAY> \\ 
number \\
<ENTRIES IN ARRAY> \\
first entry \\ 
second entry \\
... \\\
last entry }

This is a pretty barebones format. We could add a lot more meta-data to the file like data created, author program, format version, precision, variable format. For the moment we will just keep it simple.

To encapsulate interactions between the \texttt{array} and the file system we will create two functions: \texttt{arrayRead} and \texttt{arrayWrite} to read and write arrays respectively.

A barebones \texttt{arrayRead} function is as follows

\begin{minted}{c}
array arrayRead(const char *fileName){

  array a;

  FILE *fp = fopen(fileName, "r");
  if(fp){
    char buffer[BUFSIZ];
    /* skip header line */
    fgets(buffer, BUFSIZ, fp);

    /* read next line */
    int Ndata;
    fgets(buffer, BUFSIZ, fp);
    sscanf(buffer, "%d", &Ndata);

    /* construct array */
    a = arrayConstructor(Ndata);

    /* skip header line */
    fgets(buffer, BUFSIZ, fp);

    /* read array entries */
    int n, m;
    for(n=0;n<a.Ndata;n=n+1){
      fscanf(fp, "%d", &m);
      arraySet(a, n, m); /* a[n] = m */
    }
    fclose(fp);
  }
  else{
    a = arrayConstructor(0);
  }
  return a;
}
\end{minted}
The only new thing here is the use of the \texttt{fgets} to read a line of up to \texttt{BUFSIZ} characters into a stack string arrray called \texttt{buffer}. The \texttt{fgets} will read up \texttt{BUFSIZ} characters, finishing early if it encounters a line return in the file. If we were being super careful we would guard each call to \texttt{fgets} and verify that it does not return a NULL pointer. We would also check that the header line for each section contains the expected text.

A simplified implementation of \texttt{arrayWrite} follows

\begin{minted}{c}
void arrayWrite(const char *fileName, array a){

  FILE *fp = fopen(fileName, "w");

  if(fp){

    fprintf(fp, "<NUMBER OF ENTRIES IN ARRAY>\n");
    fprintf(fp, "%d\n", a.Ndata);

    fprintf(fp, "<ENTRIES IN ARRAY>\n");
    /* read array entries */
    int n, m;
    for(n=0;n<a.Ndata;n=n+1){
      int m = arrayGet(a, n); /* m = a[n] */
      fprintf(fp, "%d\n", m);
    }
    fclose(fp);
  }
  else{
    fprintf(stderr, "WARNING: failed to open %s to write\n", fileName);
  }
}
\end{minted}
This implementation applies concepts previously discussed. 

In both \texttt{arrayRead} and \texttt{arrayWrite} we follow the spirit of the \texttt{array} implementation and use the constructor, get, and set functions. Reflecting on this code we should also add a function to the growing collection of \texttt{array} functions to query the \texttt{Ndata} i.e. the number of entries in the \texttt{array}.

The following illustrates how we can read and then write an \texttt{array} file.

\begin{minted}{c}
int main(int argc, char **argv){

  array a;

  /* read array */
  a = arrayRead("arrayA.dat");

  /* write array out to copy */
  arrayWrite("copyArrayA.dat", a);

  return 0;
}
\end{minted}

We do not directly access member variables of the \texttt{array} struct at any point in this high level code.

\section{Summary of useful C standard input-output functions}

Table \ref{stdInputOutputFunctions.tab} summarizes useful standard library function calls for interacting with the heap. Table \ref{stdInputOutputDefines.tab} summarizes useful variables defined in the standard input-output library. 

\begin{table}[htbp!]
    \centering
    \begin{tabular}{l|l} \hline
      Action & Action implemented using standard input-output functions\\ \hline
        Open file to read&  \texttt{FILE* [FILE POINTER] =  fopen("[FILE NAME]", "r"];} \\
         Open file to write&  \texttt{FILE* [FILE POINTER] =  fopen("[FILE NAME]", "w"];} \\
         Formatted print to file &  \texttt{fprintf([FILE POINTER], 
         "[FORMAT]", [VAL1], [VAL2], ...);} \\
         Formatted read from file &  \texttt{fscanf([FILE POINTER], 
         "[FORMAT]", \&[VAL1], \&[VAL2], ...);} \\
         Read string from file & \texttt{fgets([BUFFER POINTER], [NUMBER OF CHARS], [FILE POINTER]);} \\
         Write a string to file & \texttt{fputs([BUFFER POINTER], [FILE POINTER]);} \\
         Flush pending writes to file & \texttt{fflush([FILE POINTER]);} \\
         Close file & \texttt{fclose([FILE POINTER]);} \\
    \hline\end{tabular}
    \caption{Summary of commonly used standard library heap functions. Here \texttt{[N]} is the number of entries of type \texttt{[TYPE]} in the arrays.}    
    \label{stdInputOutputFunctions.tab}
\end{table}

\begin{table}[htbp!]
    \centering
    \begin{tabular}{l|l} \hline
      Name & Variable\\ \hline
       Standard output terminal stream&  \texttt{stdout} \\
       Standard input terminal stream &  \texttt{stdin} \\
       Standard error terminal stream &  \texttt{stderr} \\
       Recommended buffer size  &  \texttt{BUFSIZ} \\
    \hline\end{tabular}
    \caption{Summary of standard input-output defines.}    
    \label{stdInputOutputDefines.tab}
\end{table}
