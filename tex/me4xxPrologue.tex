
%\minitoc

%%\thispagestyle{empty}
As data scientists it is quite likely that you will rely heavily on cloud computing during your career\footnote{Don't believe me ? then read this random internet  \href{https://spectrum.ieee.org/the-institute/ieee-member-news/software-engineering-grads-lack-the-skills-startups-need}{article} or this \href{https://towardsdatascience.com/why-should-every-engineer-consider-start-developing-analytical-and-programming-skills-d510eadb146c}{opinion piece}.}. 

The initial goals of this course are to introduce you to computing using a Linux terminal as a proxy for cloud computing and coding in C as a proxy for the broader family of C based languages. We choose C for its relative simplicity, for performance reasons, and as a platform to explore parallel programming concepts with OpenMP, the Message Passing Interface (MPI), and to introduce accelerated computing on graphics processing units (GPUs) with CUDA.

Cloud computing services like Amazon AWS, Google Cloud, and even Microsoft Azure typically host compute instances running Linux (and Windows to a much lesser extent). These are often accessed through a terminal with text based interactions. All of these cloud providers offer multi-core CPU and massively parallel GPU accelerated server instances with enough compute power for demanding machine learning and general data analysis related tasks.

To help motivate this approach: during an internship you might be expected to launch a cloud server instance, fire up a terminal, remotely log on to the server, and perform your work using a data analytics toolchain composed of several Linux programs that run on the server in a warehouse thousands of miles from your office. 

We will conduct guided hands on labs in class, some for credit, that cover how to use Linux and how to get you started on coding in C. However, we also encourage you to practice using the terminal and coding outside of class, i.e. try coding beyond class assignments.

After about 13 lectures we will transition from using a Linux instance hosted on your laptop by the VirtualBox  software to remotely accessing Virginia Tech servers run by the Advanced Research Computing group \href{https://www.arc.vt.edu}{https://www.arc.vt.edu}. These servers are accessed via secure shell and you will interact with them by issuing text based commands. This process is very much like using a cloud service with only a few key differences.

We will not be using canned edu-tools for this class. No Scratch, no GreenFoot, no artificial sandbox, \ldots. We will only use tools that are open source, used for real software development. Unfortunately this means that they let you make mistakes. You will need to pay attention to how the tools are used and even use your favorite search engine to find help on how to use them in some cases. We hope that learning with these tools in this course will pay off in your career. 

\begin{center}
    \underline{Let the text based adventure begin !}
\end{center}