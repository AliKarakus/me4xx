
    \hypertarget{how-are-numpy-arrays-stored}{%
\subsection{How are numpy arrays
stored?}\label{how-are-numpy-arrays-stored}}

    \begin{BGVerbatim}[commandchars=\\\{\}]
{\color{incolor}In [{\color{incolor}1}]:} \PY{k+kn}{import} \PY{n+nn}{numpy} \PY{k}{as} \PY{n+nn}{np}
\end{BGVerbatim}

    Numpy presents an n-dimensional abstraction that has to be fit into
1-dimensional computer memory.

Even for 2 dimensions (matrices), this leads to confusion: row-major,
column-major.

    \begin{BGVerbatim}[commandchars=\\\{\}]
{\color{incolor}In [{\color{incolor}2}]:} \PY{n}{A} \PY{o}{=} \PY{n}{np}\PY{o}{.}\PY{n}{arange}\PY{p}{(}\PY{l+m+mi}{9}\PY{p}{)}\PY{o}{.}\PY{n}{reshape}\PY{p}{(}\PY{l+m+mi}{3}\PY{p}{,} \PY{l+m+mi}{3}\PY{p}{)}
        \PY{n+nb}{print}\PY{p}{(}\PY{n}{A}\PY{p}{)}
\end{BGVerbatim}

    \begin{BGVerbatim}[commandchars=\\\{\}]
[[0 1 2]
 [3 4 5]
 [6 7 8]]

    \end{BGVerbatim}

    \hypertarget{strides-and-in-memory-representation}{%
\subsubsection{Strides and in-memory
representation}\label{strides-and-in-memory-representation}}

    How is this represented in memory?

    \begin{BGVerbatim}[commandchars=\\\{\}]
{\color{incolor}In [{\color{incolor}3}]:} \PY{n}{A}\PY{o}{.}\PY{n}{strides}
\end{BGVerbatim}

\begin{BGVerbatim}[commandchars=\\\{\}]
{\color{outcolor}Out[{\color{outcolor}3}]:} (24, 8)
\end{BGVerbatim}
            
    \begin{itemize}
\tightlist
\item
  \texttt{strides} stores for each axis by how many bytes one needs to
  jump to get from one entry to the next (in that axis)
\item
  So how is the array above stored?
\item
  This captures row-major (``C'' order) and column-major (``Fortran''
  order), but is actually much more general.
\end{itemize}

    \begin{BGVerbatim}[commandchars=\\\{\}]
{\color{incolor}In [{\color{incolor}10}]:} \PY{n+nb}{bytes}\PY{p}{(}\PY{n}{A}\PY{o}{.}\PY{n}{astype}\PY{p}{(}\PY{n}{np}\PY{o}{.}\PY{n}{uint8}\PY{p}{)}\PY{p}{)}
\end{BGVerbatim}

\begin{BGVerbatim}[commandchars=\\\{\}]
{\color{outcolor}Out[{\color{outcolor}10}]:} b'\textbackslash{}x00\textbackslash{}x01\textbackslash{}x02\textbackslash{}x03\textbackslash{}x04\textbackslash{}x05\textbackslash{}x06\textbackslash{}x07\textbackslash{}x08'
\end{BGVerbatim}
            
    We can also ask for Fortran order:

    \begin{BGVerbatim}[commandchars=\\\{\}]
{\color{incolor}In [{\color{incolor}11}]:} \PY{n}{A2} \PY{o}{=} \PY{n}{np}\PY{o}{.}\PY{n}{arange}\PY{p}{(}\PY{l+m+mi}{9}\PY{p}{)}\PY{o}{.}\PY{n}{reshape}\PY{p}{(}\PY{l+m+mi}{3}\PY{p}{,} \PY{l+m+mi}{3}\PY{p}{,} \PY{n}{order}\PY{o}{=}\PY{l+s+s2}{\PYZdq{}}\PY{l+s+s2}{F}\PY{l+s+s2}{\PYZdq{}}\PY{p}{)}
         \PY{n}{A2}
\end{BGVerbatim}

\begin{BGVerbatim}[commandchars=\\\{\}]
{\color{outcolor}Out[{\color{outcolor}11}]:} array([[0, 3, 6],
                [1, 4, 7],
                [2, 5, 8]])
\end{BGVerbatim}
            
    \texttt{numpy} defaults to row-major order.

    \begin{BGVerbatim}[commandchars=\\\{\}]
{\color{incolor}In [{\color{incolor}12}]:} \PY{n}{A2}\PY{o}{.}\PY{n}{strides}
\end{BGVerbatim}

\begin{BGVerbatim}[commandchars=\\\{\}]
{\color{outcolor}Out[{\color{outcolor}12}]:} (8, 24)
\end{BGVerbatim}
            
    \begin{BGVerbatim}[commandchars=\\\{\}]
{\color{incolor}In [{\color{incolor}13}]:} \PY{n+nb}{bytes}\PY{p}{(}\PY{n}{A2}\PY{o}{.}\PY{n}{astype}\PY{p}{(}\PY{n}{np}\PY{o}{.}\PY{n}{uint8}\PY{p}{)}\PY{p}{)}
\end{BGVerbatim}

\begin{BGVerbatim}[commandchars=\\\{\}]
{\color{outcolor}Out[{\color{outcolor}13}]:} b'\textbackslash{}x00\textbackslash{}x03\textbackslash{}x06\textbackslash{}x01\textbackslash{}x04\textbackslash{}x07\textbackslash{}x02\textbackslash{}x05\textbackslash{}x08'
\end{BGVerbatim}
            
    \hypertarget{strides-and-contiguity}{%
\subsubsection{Strides and Contiguity}\label{strides-and-contiguity}}

    How is the stride model more general than just saying ``row major'' or
``column major''?

    \begin{BGVerbatim}[commandchars=\\\{\}]
{\color{incolor}In [{\color{incolor}14}]:} \PY{n}{A} \PY{o}{=} \PY{n}{np}\PY{o}{.}\PY{n}{arange}\PY{p}{(}\PY{l+m+mi}{16}\PY{p}{)}\PY{o}{.}\PY{n}{reshape}\PY{p}{(}\PY{l+m+mi}{4}\PY{p}{,} \PY{l+m+mi}{4}\PY{p}{)}
         \PY{n}{A}
\end{BGVerbatim}

\begin{BGVerbatim}[commandchars=\\\{\}]
{\color{outcolor}Out[{\color{outcolor}14}]:} array([[ 0,  1,  2,  3],
                [ 4,  5,  6,  7],
                [ 8,  9, 10, 11],
                [12, 13, 14, 15]])
\end{BGVerbatim}
            
    \begin{BGVerbatim}[commandchars=\\\{\}]
{\color{incolor}In [{\color{incolor}15}]:} \PY{n}{A}\PY{o}{.}\PY{n}{strides}
\end{BGVerbatim}

\begin{BGVerbatim}[commandchars=\\\{\}]
{\color{outcolor}Out[{\color{outcolor}15}]:} (32, 8)
\end{BGVerbatim}
            
    \begin{BGVerbatim}[commandchars=\\\{\}]
{\color{incolor}In [{\color{incolor}16}]:} \PY{n}{Asub} \PY{o}{=} \PY{n}{A}\PY{p}{[}\PY{p}{:}\PY{l+m+mi}{3}\PY{p}{,} \PY{p}{:}\PY{l+m+mi}{3}\PY{p}{]}
         \PY{n}{Asub}
\end{BGVerbatim}

\begin{BGVerbatim}[commandchars=\\\{\}]
{\color{outcolor}Out[{\color{outcolor}16}]:} array([[ 0,  1,  2],
                [ 4,  5,  6],
                [ 8,  9, 10]])
\end{BGVerbatim}
            
    Recall that \texttt{Asub} constitutes a \emph{view} of the original data
in \texttt{A}.

    \begin{BGVerbatim}[commandchars=\\\{\}]
{\color{incolor}In [{\color{incolor}17}]:} \PY{n}{Asub}\PY{o}{.}\PY{n}{strides}
\end{BGVerbatim}

\begin{BGVerbatim}[commandchars=\\\{\}]
{\color{outcolor}Out[{\color{outcolor}17}]:} (32, 8)
\end{BGVerbatim}
            
    Now \texttt{Asub} is no longer a \emph{contiguous} array!

From the linear-memory representation (as show by the increasing numbers
in \texttt{A}) 3, 7, 11 are missing.

This is easy to check by a flag:

    \begin{BGVerbatim}[commandchars=\\\{\}]
{\color{incolor}In [{\color{incolor}18}]:} \PY{n}{Asub}\PY{o}{.}\PY{n}{flags}
\end{BGVerbatim}

\begin{BGVerbatim}[commandchars=\\\{\}]
{\color{outcolor}Out[{\color{outcolor}18}]:}   C\_CONTIGUOUS : False
           F\_CONTIGUOUS : False
           OWNDATA : False
           WRITEABLE : True
           ALIGNED : True
           WRITEBACKIFCOPY : False
           UPDATEIFCOPY : False
\end{BGVerbatim}
            
    \begin{BGVerbatim}[commandchars=\\\{\}]
{\color{incolor}In [{\color{incolor} }]:} 
\end{BGVerbatim}


