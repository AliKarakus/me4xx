\chapterauthor{Anthony P. Austin}

\minitoc

\section{Introduction}

In the previous lecture, you created a Git repository for use with this class
on GitLab, cloned it on your virtual machine, and learned some basic Git
commands.  Here, we take a closer look at these basic commands and add several
others to our repertoire.

\section{Basic Git Workflow:  Edit--Add--Commit}

The basic workflow for using Git to track changes to your files is one of
\emph{edit}--\emph{add}--\emph{commit}:

\begin{enumerate}

\item \emph{Edit} your files.

\item \emph{Add} the changes to Git's \emph{index}.

\item \emph{Commit} the changes staged in the index to the repository.

\end{enumerate}

\begin{tsession}{mytermbg}
\begin{verbatim}
$ emacs file.txt
$ git add file.txt
$ git commit
\end{verbatim}
\end{tsession}

Forgetting step 2 (``add'') is a common pitfall, especially for people used to
version control systems like CVS and Subversion, which have no analogue.  It is
good practice to use \texttt{git status} before committing to make sure you
have added all desired changes to the index:

\begin{tsession}{mytermbg}
\begin{verbatim}
$ git status
On branch master
Changes to be committed:
  (use "git reset HEAD <file>..." to unstage)

        modified:   file.txt

Changes not staged for commit:
  (use "git add <file>..." to update what will be committed)
  (use "git checkout -- <file>..." to discard changes in working directory)

        modified:   anotherFile.txt
\end{verbatim}
\end{tsession}

This output shows that our changes to \texttt{file.txt} have been added to the
index but that those to \texttt{anotherFile.txt} have not.  Thus, if we next
run \texttt{git commit}, the changes to \texttt{file.txt} will be committed,
but those to \texttt{anotherFile.txt} will not.

If you add changes to the index and later decide that you do not want them to
be part of your commit, you can un-stage them with \texttt{git reset}.
Continuing the example from above, we can unstage the changes to
\texttt{file.txt} as follows:

\begin{tsession}{mytermbg}
\begin{verbatim}
$ git reset file.txt
Unstaged changes after reset:
M	anotherFile.txt
M	file.txt
\end{verbatim}
\end{tsession}

(The \texttt{M} indicates that the file was modified.)  Now \texttt{git status} shows

\begin{tsession}{mytermbg}
\begin{verbatim}
$ git status
On branch master
Changes not staged for commit:
  (use "git add <file>..." to update what will be committed)
  (use "git checkout -- <file>..." to discard changes in working directory)

	modified:   anotherFile.txt
	modified:   file.txt

no changes added to commit (use "git add" and/or "git commit -a")
\end{verbatim}
\end{tsession}

Both files have been modified, but changes to neither have been staged for
commit.

To see what changes have been made since the last commit, you can use
\texttt{git diff}:

\begin{tsession}{mytermbg}
\begin{verbatim}
$ git diff
diff --git a/anotherFile.txt b/anotherFile.txt
index 9ca9a87..7603f14 100644
--- a/anotherFile.txt
+++ b/anotherFile.txt
@@ -1 +1,2 @@
 This is another file.
+This is a new line in the other file.
diff --git a/file.txt b/file.txt
index 4dd1ef7..685e488 100644
--- a/file.txt
+++ b/file.txt
@@ -1 +1,2 @@
 This is a file.
+This is a new line in the file.
\end{verbatim}
\end{tsession}

This outputs the differences in the files in the unified diff format.  Lines
beginning with a \texttt{+} have been added since the last commit; lines
beginning with a \texttt{-} have been deleted.  Note that \texttt{git diff}
only shows changes that have not been staged (added to the index).  To see
staged changes, use \texttt{git diff --cached}:

\begin{tsession}{mytermbg}
\begin{verbatim}
$ git add file.txt
$ git diff
diff --git a/anotherFile.txt b/anotherFile.txt
index 9ca9a87..9b6f183 100644
--- a/anotherFile.txt
+++ b/anotherFile.txt
@@ -1 +1 @@
-This is another file.
+This is a new line in the other file that replaces the old one.
$ git diff --cached
diff --git a/file.txt b/file.txt
index 4dd1ef7..685e488 100644
--- a/file.txt
+++ b/file.txt
@@ -1 +1,2 @@
 This is a file.
+This is a new line in the file.
\end{verbatim}
\end{tsession}

If you've changed a file but decide that you do not want to keep the changes,
you can use \texttt{git checkout} to discard them:

\begin{tsession}{mytermbg}
\begin{verbatim}
$ git checkout anotherFile.txt
$ git status
On branch master
Changes to be committed:
  (use "git reset HEAD <file>..." to unstage)

	modified:   file.txt
\end{verbatim}
\end{tsession}

Notice that \texttt{git status} no longer lists \texttt{anotherFile.txt} as
modified.

When you run \texttt{git commit}, Git will open a text editor (usually
\texttt{vi} by default) to let you write your commit message.  The standard
format for a Git commit message is to begin with a single short (less than
80-character) line summarizing the changes in the commit, followed by a blank
line, and then more thorough description as needed.  Get in the habit of
writing good, informative commit messages.  Your collaborators (including your
future self) will thank you, as good commit messages can make it much easier to
trace the history of development of a piece of code by revealing what the
developer was thinking when the changes were checked in.

\section{Looking Through History}

Git has a number of commands for viewing the commit history of a repository.
The most basic of these is \texttt{git log}:

\begin{tsession}{mytermbg}
\begin{verbatim}
$ git log
commit cae7fa52ce9d58aba4f743ff15f6fafb684da4d6
Author: J. D. Hokie <jdh@vt.edu>
Date:   Tue Sep 3 20:23:07 2019 -0400

    Add numbered headers to files.

commit 9759687249220a7e079eeab185c9f10c76775351
Author: J. D. Hokie <jdh@vt.edu>
Date:   Tue Sep 3 20:22:35 2019 -0400

    Add yet another file.

commit 16db20ec863d28215fd9532133bbf6d35234b604
Author: J. D. Hokie <jdh@vt.edu>
Date:   Tue Sep 3 20:21:20 2019 -0400

    Add a new line to the first file.

commit ffef1cf6368a65aec94ff34da53369e32e663191
Author: J. D. Hokie <jdh@vt.edu>
Date:   Tue Sep 3 16:47:31 2019 -0400

    Add another file.

commit 0bd6c8e9f7370881de6526db9745c54eefbd2caf
Author: J. D. Hokie <jdh@vt.edu>
Date:   Tue Sep 3 15:37:53 2019 -0400

    Initial commit.

\end{verbatim}
\end{tsession}

The history begins with the most recent commit and proceeds backward in time.
Each entry shows the commit author's name and e-mail address, a timestamp with
the date of creation, and the message that the author wrote when the commit was
made.

The long hexadecimal string at the beginning of each log entry is a 160-bit
SHA-1 hash that uniquely identifies the commit.  These hashes can be used
throughout Git to refer to commits.  You can abbreviate a hash to its first few
characters if that is enough to identify it.  As an example, if you want to see
the full log entry, including the patch, associated with the third commit above
(\texttt{16db20ec}), you can do

\begin{tsession}{mytermbg}
\begin{verbatim}
$ git show 16db20ec
commit 16db20ec863d28215fd9532133bbf6d35234b604
Author: J. D. Hokie <jdh@vt.edu>
Date:   Tue Sep 3 20:21:20 2019 -0400

    Add a new line to the first file.

diff --git a/file.txt b/file.txt
index 4dd1ef7..685e488 100644
--- a/file.txt
+++ b/file.txt
@@ -1 +1,2 @@
 This is a file.
+This is a new line in the file.
\end{verbatim}
\end{tsession}

The \texttt{git show} command can also show the state of a file at the time of
a given commit, e.g.:

\begin{tsession}{mytermbg}
\begin{verbatim}
$ git show 16db20ec:file.txt
\end{verbatim}
\end{tsession}

Just as \texttt{git diff} can be used to view changes made since the previous
commit, it can also be used to show changes made between two commits.  For
instance, to see the changes between \texttt{16db20ec} and the current commit
you can do:

\begin{tsession}{mytermbg}
\begin{verbatim}
$ git diff 16db20ec..HEAD
diff --git a/anotherFile.txt b/anotherFile.txt
index 9ca9a87..ece730d 100644
--- a/anotherFile.txt
+++ b/anotherFile.txt
@@ -1 +1,3 @@
+FILE 2
+------
 This is another file.
diff --git a/file.txt b/file.txt
index 685e488..a89ec53 100644
--- a/file.txt
+++ b/file.txt
@@ -1,2 +1,4 @@
+FILE 1
+------
 This is a file.
 This is a new line in the file.
diff --git a/yetAnotherFile.txt b/yetAnotherFile.txt
new file mode 100644
index 0000000..a4a93e0
--- /dev/null
+++ b/yetAnotherFile.txt
@@ -0,0 +1,3 @@
+FILE 3
+------
+This is yet another file.
\end{verbatim}
\end{tsession}

The identifier \texttt{HEAD} is shorthand for the commit at the head of the
currently checked-out branch (see the section on branching below).

Git's command-line tools for navigating history are useful, but it can also be
nice to have a way to view history graphically, especially when the history
branches frequently.  Git ships with a graphical history viewer written in Tcl
called \texttt{gitk}.

\boximage{figures/L04/gitk.png}{The \texttt{gitk} graphical Git history viewer.}

Each commit on the current branch is displayed with the \texttt{HEAD} commit at
the top.  The text next to each commit in the graphical history window is the
first line from the commit message.  Below the history window, one can see the
currently-selected commit's SHA-1, and below that, the full commit message and
diff, along with a list of the files that have been modified.

Note that \texttt{gitk} is \emph{not} a graphical interface to Git!  Such
interfaces do exist, but we discourage their use.  Git was designed to be used
from the command line, and that is the most popular way to use it.  Many of
Git's more advanced features are only accessible via the command line.  The
command line interface to Git has the added advantage of being scriptable,
enabling easy automation of development tasks in ways that graphical interfaces
do not support.

\section{Sharing Changes}

Unless you work in isolation, at some point, you will need to share the changes
you make to a codebase with others.  You will also need to be able to receive
receive changes others have made and incorporate them into your own copy of the
code.  Git's features for sharing changes to code are flexible and powerful and
are one of the two major reasons Git has taken the world of open-source
development by storm (the other being its facilities for branching, which we
introduce briefly in the next section).

In this course, we will stick to the basics.  You will share your code with the
course instructor and teaching assistants by putting them in the repository you
created on VT CodeLab.  To get code from your laptop to CodeLab, the basic
command is \texttt{git push}:

\begin{tsession}{mytermbg}
\begin{verbatim}
$ git push
Counting objects: 3, done.
Delta compression using up to 8 threads.
Compressing objects: 100% (3/3), done.
Writing objects: 100% (3/3), 359 bytes | 359.00 KiB/s, done.
Total 3 (delta 0), reused 0 (delta 0)
To git@code.vt.edu:jdh/CMDA3634.git
   4041187..d501a5f  master -> master
\end{verbatim}
\end{tsession}

Later in the course, you will also need to know how to get changes from CodeLab
to your laptop.  The basic command for this is \texttt{git pull}:

\begin{tsession}{mytermbg}
\begin{verbatim}
$ git pull
remote: Counting objects: 3, done.
remote: Compressing objects: 100% (3/3), done.
remote: Total 3 (delta 0), reused 0 (delta 0)
Unpacking objects: 100% (3/3), done.
From git@code.vt.edu:jdh/CMDA3634.git
 * branch            HEAD       -> FETCH_HEAD
Updating 4041187..d501a5f
Fast-forward
 anotherFile.txt | 1 +
 1 file changed, 1 insertion(+)
\end{verbatim}
\end{tsession}

Note that you cannot push new changes to a repository unless you have already
pulled the latest version of the files being tracked from it already.  If you
forget to do this, and if the repository to which you are trying to push your
changes has other changes that you do not yet have, you will get an error like
this:

\begin{tsession}{mytermbg}
\begin{verbatim}
$ git push
To git@code.vt.edu:jdh/CMDA3634.git
 ! [rejected]        master -> master (fetch first)
error: failed to push some refs to 'git@code.vt.edu:jdh/CMDA3634.git'
hint: Updates were rejected because the remote contains work that you do
hint: not have locally. This is usually caused by another repository pushing
hint: to the same ref. You may want to first integrate the remote changes
hint: (e.g., 'git pull ...') before pushing again.
hint: See the 'Note about fast-forwards' in 'git push --help' for details.
\end{verbatim}
\end{tsession}

To fix this, do \texttt{git pull} first; then retry \texttt{git push}:

\begin{tsession}{mytermbg}
\begin{verbatim}
$ git pull
remote: Counting objects: 3, done.
remote: Compressing objects: 100% (3/3), done.
remote: Total 3 (delta 1), reused 0 (delta 0)
Unpacking objects: 100% (3/3), done.
From git@code.vt.edu:jdh/CMDA3634.git
   d501a5f..b932eba  master     -> origin/master
Merge made by the 'recursive' strategy.
 yetAnotherFile.txt | 1 +
 1 file changed, 1 insertion(+)
$ git push
Counting objects: 5, done.
Delta compression using up to 8 threads.
Compressing objects: 100% (5/5), done.
Writing objects: 100% (5/5), 580 bytes | 580.00 KiB/s, done.
Total 5 (delta 2), reused 0 (delta 0)
To git@code.vt.edu:jdh/CMDA3634.git
   b932eba..3a51a4e  master -> master
\end{verbatim}
\end{tsession}

You may be prompted for a commit message for a merge commit.

Notice that in neither of these commands did we specify where to push our data
or from where to pull it.  The reason is that Git remembers the URL of your
repository on CodeLab from when you cloned it in the previous lecture.  From
the perspective of the repository on your laptop, your repository on CodeLab
repository is what Git calls a \emph{remote repository}, or just a
\emph{remote} for short.  To see a list of all the remotes of which your
current repository is aware, you can use \texttt{git remote -v}:

\begin{tsession}{mytermbg}
\begin{verbatim}
$ git remote -v
origin	git@code.vt.edu:jdh/CMDA3634.git (fetch)
origin	git@code.vt.edu:jdh/CMDA3634.git (push)
\end{verbatim}
\end{tsession}

This output shows that you have a remote named \texttt{origin} that refers to
your repository on CodeLab.   The two URLs shown are the locations from which
Git will retrieve (``fetch'') changes and to which Git will send (``push'')
changes when interacting with this remote.  The two URLs are almost always the
same; in rare circumstances, it can be beneficial for them to be different.

The \texttt{git push} command shown above is actually shorthand for

\begin{tsession}{mytermbg}
\begin{verbatim}
$ git push origin master
\end{verbatim}
\end{tsession}

which instructs Git to push commits from the \texttt{master} branch on your
laptop's repository to the \texttt{master} branch of the repository specified
by the \texttt{origin} remote, i.e., your CodeLab repository.  Likewise

\begin{tsession}{mytermbg}
\begin{verbatim}
$ git pull origin master
\end{verbatim}
\end{tsession}

tells git to pull changes from the \texttt{master} branch on CodeLab into the
currently checked-out branch on your laptop.

It is considered good practice to use the long forms of these commands,
specifying both the remote and the branch name explicitly.  This greatly
reduces the likelihood of accidentally sending (receiving) changes to (from)
the wrong place when you work with multiple remotes.

\section{Branching Basics}

What do we mean by the \texttt{master} branch?  In the language of version
control systems, a \emph{branch} refers to a line of code development.  In this
class, you have only one line of development to worry about---that associated
with your own coursework---but in large projects with multiple developers, it
is common to have tens, hundreds, or even thousands of lines of development
that are active at once on the same codebase.  Modern version control systems
all have facilities for supporting multiple branches.

In Git, branches are particularly simple:  they're essentially just references
to specific commits.  A branch ``points'' to a commit, and when a new commit is
made ``to'' that branch, two things happen.  First, the commit to which the
branch currently refers is made the immediate ancestor of the new commit.
Second, the branch is updated to point to the new commit.

When a Git repository is first initialized, it comes with a default branch
named \texttt{master} that is set to the current branch.  Many (but not all)
software projects that use Git for version control retain this branch and use
it to refer to the most recently released or ``stable'' version of the software
or to code that will be in the next release.

As mentioned above, in this class, you have only one line of development, and
you will be making all of your commits to your repository's \texttt{master}
branch, but as real-world codebases frequently have multiple branches, it is
useful to know the basics for how to work with branches in Git.

You can see what branch you are currently on with \texttt{git status}:

\begin{tsession}{mytermbg}
\begin{verbatim}
$ git status
On branch master
nothing to commit, working tree clean
\end{verbatim}
\end{tsession}

The output clearly states that we are on the \texttt{master} branch.

You can see a list of all the branches in your repository with \texttt{git
branch}

\begin{tsession}{mytermbg}
\begin{verbatim}
$ git branch
  experimental
* master
\end{verbatim}
\end{tsession}

This repository has two branches:  \texttt{experimental} and \texttt{master}.
The current branch is indicated by an asterisk (\texttt{*}).

The branch that's currently active is said to be ``checked out''.  To check out
a different branch, use \texttt{git checkout}:

\begin{tsession}{mytermbg}
\begin{verbatim}
$ git checkout experimental
Switched to branch 'experimental'
\end{verbatim}
\end{tsession}

When you do this, the files in your repository will automatically change to
whatever state they were in on the commit at the head of that branch.

You can create a new branch and check it out with \texttt{git checkout -b}:

\begin{tsession}{mytermbg}
\begin{verbatim}
$ git checkout -b newBranch
Switched to a new branch 'newBranch'
\end{verbatim}
\end{tsession}

The head of the new branch will be whatever commit is currently checked out.
If you want it to be somewhere else, you can specify it by its SHA-1:

\begin{tsession}{mytermbg}
\begin{verbatim}
$ git checkout -b anotherBranch 40411875
Switched to a new branch 'anotherBranch'
\end{verbatim}
\end{tsession}

To merge changes from one branch into another, the command is \texttt{git
merge}.  This command merges changes from a given branch into the branch that
is currently checked out.  To merge changes from a branch named
\texttt{experimental} into the \texttt{master} branch, you do:

\begin{tsession}{mytermbg}
\begin{verbatim}
$ git checkout master
Switched to branch 'master'
$ git merge experimental
Updating 3a3637a..ee3580a
Fast-forward
 file.txt | 1 +
 1 file changed, 1 insertion(+)
\end{verbatim}
\end{tsession}

In this case, the output of \texttt{git merge} indicated that this was a
\emph{fast-forward} merge.  This means that the head of the branch named in
\texttt{git merge} was a descendent of the checked-out branch into which the
merge was made.  In this case, the merge operation is trivial:  Git just
redefines the \texttt{master} branch to point to the same commit as the
\texttt{experimental} one.

For merges that are not fast-forward merges, one must make a \emph{merge
commit} to join the history of the two branches.  When this happens, Git will
prompt you for a commit message.  The body of the message will already be
filled in; you can replace it if you choose.

What happens if both branches have commits that modify the same file in
different ways?  In this scenario, Git will try to figure out whether the
changes clash with one another and merge them automatically.  If it cannot do
this, you will get a message indicating that the changes conflict:

\begin{tsession}{mytermbg}
\begin{verbatim}
$ git merge experimental
Auto-merging file.txt
CONFLICT (content): Merge conflict in file.txt
Automatic merge failed; fix conflicts and then commit the result.
\end{verbatim}
\end{tsession}

In this case, you will need to edit the files with conflicting changes, add the
changes to the index with \texttt{git add}, and then use \texttt{git commit} to
complete the merge.  Git will mark the parts of the files showing where there
were conflicts that it could not resolve.

\section{An Extended Example}

The following is an example of how Git is used in a typical real-world
development scenario.  Your work in this class will follow a similar workflow
except that as you are working on your own, you can do all your work in a
single line of development without branches (or, rather, using only the
\texttt{master} branch) if you choose.

Alice's boss has asked her to implement a new feature in their codebase.  As
the new feature is likely to be unstable, she's been asked to do her
development in a new branch to keep it separate from the main codebase being
tracked by the project's \texttt{master} branch.  Alice creates a new branch
based off of \texttt{master} and does some work:

\begin{tsession}{mytermbg}
\begin{verbatim}
$ git checkout -b new-feature master
$ emacs newFeature.txt
$ git add newFeature.txt
$ git commit
\end{verbatim}
\end{tsession}

Alice reports back to her boss, saying that the new feature is finished.  Her
boss says that before it can be incorporated into the stable \texttt{master}
branch, it needs to undergo review by the project lead.  Alice pushes her
branch to the project's central repository so the lead can see it.

\begin{tsession}{mytermbg}
\begin{verbatim}
$ git push origin new-feature
\end{verbatim}
\end{tsession}

The project lead makes a few small changes and then asks Alice to make some
others.  Alice first makes sure she's on the branch for her feature and then
pulls the lead's changes into her repository.  She then does the additional
work and pushes it back for further review:

\begin{tsession}{mytermbg}
\begin{verbatim}
$ git checkout new-feature
$ git pull origin new-feature
$ emacs newFeature.txt
$ git add newFeature.txt
$ git commit
$ git push origin new-feature
\end{verbatim}
\end{tsession}

The project lead is now happy with her code and asks Alice to incorporate it
into the \texttt{master} branch.  She switches to the \texttt{master} branch,
makes sure she has pulled any updates that have been pushed to it in the
meantime and then merges her feature branch into it with \texttt{git merge}:

\begin{tsession}{mytermbg}
\begin{verbatim}
$ git checkout master
$ git pull origin master
$ git merge new-feature
\end{verbatim}
\end{tsession}

Finally, she pushes the newly updated \texttt{master} branch to her company's
central repository and deletes the local and remote versions of her feature
branch, which are no longer needed:

\begin{tsession}{mytermbg}
\begin{verbatim}
$ git push origin master
$ git push origin :new-feature
$ git branch -d new-feature
\end{verbatim}
\end{tsession}

\section{Miscellaneous Tips}

Git has many useful features---far too many to cover comprehensively in these
notes.  Here are a few additional features that may be useful to you in this
course.

\subsection{Ignoring Files}

Often, you have files that you do not want Git to track.  You can get Git to
ignore these files by creating a file named \texttt{.gitignore} in your
repository and adding the files that you do not want tracked to it, one per
line.  Wildcard patterns are accepted; see  \texttt{man gitignore} for details
about the syntax.

As a basic example, this \texttt{.gitignore}

\begin{minted}{text}
*.o
myProgram
\end{minted}

will ignore the file \texttt{myProgram} and all object files ending in with the
generated by a compilation process.  Why track these when they can be
regenerated from the (tracked) source files?

Patterns specified in a \texttt{.gitignore} file apply to the directory in
which the \texttt{.gitignore} is located any any subdirectories.  Note that the
\texttt{.gitignore} file is trackable by Git; this is useful for distributing
project-wide ignore rules to all developers.

\subsection{Tracking Empty Directories}

Git tracks files, not directories.  You cannot add a directory to a repository
unless it has a file in it.  If you just want to set up a directory structure
and have Git track it, one way to accomplish this is to create a single empty
file in the directory, conventionally named \texttt{.KEEP} or
\texttt{.gitkeep}, and commit that file.

\subsection{Deleting, Moving, and Renaming Files}

The important thing to remember when deleting, moving, or renaming files is
that each of these operations is a change that must be added to Git's index
before it can be committed.  For instance, to delete a file, you can use
\texttt{rm} like usual, followed by \texttt{git add}:

\begin{tsession}{mytermbg}
\begin{verbatim}
$ rm file.txt
$ git add file.txt
$ git status
On branch master
Changes to be committed:
  (use "git reset HEAD <file>..." to unstage)

	deleted:    file.txt

\end{verbatim}
\end{tsession}

This two-step sequence is a little awkward---it feels weird to \texttt{git add}
a file that we've deleted---but there is a shortcut; you can use \texttt{git
rm} instead:

\begin{tsession}{mytermbg}
\begin{verbatim}
$ git rm file.txt
\end{verbatim}
\end{tsession}

Similarly, to move or rename a file, you can use \texttt{mv} followed by
\texttt{git add}, or you can use \texttt{git mv} to do both at once.

\subsection{Undoing a Commit}

If you need to undo a commit, the command to use is \texttt{git revert}.  For
example, if commit \texttt{b932eba4} turned out to be a mistake, you can do

\begin{tsession}{mytermbg}
\begin{verbatim}
$ git revert b932eba4
\end{verbatim}
\end{tsession}

This command works by creating a new commit that is the ``inverse'' of the old
commit.  As such, when you run this command, you will be prompted for a commit
message, giving you some a place to explain the need for the reversion.

\subsection{Getting Help}

If you can't remember the options for a given command and need a quick
reminder, the Git manual pages are a good place to look.  The manual pages are
named after the commands with spaces replaced by hyphens; for instance, to
retrieve the manual page for \texttt{git add}, type \texttt{man git-add}.

The top-level manual page (\texttt{man git}) has, among other things, a list of
all the Git commands, including many that we have not discussed here.  Other
manual pages of interest include the Git tutorial at \texttt{man gittutorial}
(and its sequel \texttt{man gittutorial-2}), the glossary at \texttt{man
gitglossary}, and the summary of ``everyday'' Git commands and workflows at
\texttt{man giteveryday}.

\newpage
\section{Summary of Git Commands}

Table \ref{gitCommandsFull.tab} extends the summary table of Git commands from
the previous lecture to include the new commands we have covered above.

\begin{table}[htbp!]
    \centering
    \begin{tabular}{l|l} \hline
      Action & \verb|git| commands \\ \hline
         Clone repo &  \texttt{git clone [URL OF REPO]}\\ \hline
         Check repo status & \texttt{git status} \\
         Add file change to local repo & \texttt{git add [FILE NAME]} \\
         Commit change to local repo & \texttt{git commit -m '[NOTE TO ATTACH TO COMMIT]'} \\
         Push committed changes to remote repo & \texttt{git push} \\
         Update local repo with remote changes & \texttt{git pull} \\ \hline
         Unstage a staged change & \texttt{git reset [FILE]} \\
         Discard changes to a file & \texttt{git checkout [CHANGED FILE]} \\
         View unstaged changes since the last commit & \texttt{git diff} \\
         View staged changes since the last commit & \texttt{git diff --cached} \\
         Delete a tracked file and stage the deletion & \texttt{git rm [FILE]} \\
         Move a tracked file and stage the move & \texttt{git mv [FILE] [NEW FILE]} \\
         Undo a commit & \texttt{git revert [COMMIT]} \\ \hline
         View history of the current branch & \texttt{git log} \\
         View the message and patch for a commit & \texttt{git show [COMMIT]} \\
         View a file as it was on a particular commit & \texttt{git show [COMMIT]:[FILE]} \\
         View difference between two commits & \texttt{git diff [OLD COMMIT]..[NEW COMMIT]} \\
         Launch Git's built-in graphical history viewer & \texttt{gitk} \\ \hline
         View remotes associated with local repository & \texttt{git remote -v} \\
         Push committed changes to remote repo (full form) & \texttt{git push [REMOTE] [BRANCH]} \\
         Update local repo with remote changes (full form) & \texttt{git pull [REMOTE] [BRANCH]} \\ \hline
         Create a new branch & \texttt{git branch [NAME OF BRANCH TO CREATE]} \\
         Switch to a different branch & \texttt{git checkout [NAME OF BRANCH TO SWITCH TO]} \\
         Create and switch to a different branch & \texttt{git checkout -b [NEW BRANCH] [BRANCH LOCATION]} \\ \hline
         Merge given branch into current one & \texttt{git merge [BRANCH TO MERGE]} \\ \hline
    \end{tabular}
    \caption{Summary of commonly used Git command line tool commands.}
    \label{gitCommandsFull.tab}
\end{table}

\section{Further Reading}

The first three chapters of \emph{Pro Git} by Scott Chacon and Ben Straub,
available at
\href{https://git-scm.com/book/en/v2}{https://git-scm.com/book/en/v2} contain
all of the information you need for this class and then some.  If you are
interested in learning how Git works internally, the first five sections of
Chapter 10 are also of interest.
