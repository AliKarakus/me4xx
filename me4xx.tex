\documentclass{book}[11pt]
%\documentclass[notoc]{tufte-book} 
\usepackage{caption}
\usepackage{subcaption}
%% -------------------------------------------
%% PREAMBLE STARTS HERE 
\input preamble
\input notebookPreamble

%% document title
\title{Computational Science Foundations for \\Mechanical Engineers\\ ME 4XX \\ \vspace{16pt}Lecture Notes \\ Spring 2020}

%% PREAMBLE ENDS HERE
%% -------------------------------------------

\begin{document}
\setlength{\parindent}{0cm}

\maketitle

\copyrightPage

\mytoc

\section*{Foreword}
\input tex/me4xxForeword

\newpage
\section*{About this course}
\input tex/me4xxPrologue


\newpage
\section*{Lecture plan}
\input tex/me4xxLecturePlan

\part{Intro. computational science skills}

\chapter{Setting up an Ubuntu Linux VirtualBox}
%\renewcommand\thesection{\thechapter.\arabic{section}}
\label{linux.chap}
\adjustmtc[0]
\input tex/me4xxL01

\chapter{Linux Command Terminal}
\label{terminal.chap}
\input tex/me4xxL02

\chapter{Setting up a Git repository}
\label{git.chap}
\input tex/me4xxL03

\chapter{Source code control using Git}
\label{gitSource.chap}
\input tex/me4xxL04

% \chapter{Typesetting technical documents with \LaTeX{}}
% \label{latex.chap}
% \input tex/me4xxL05

% \chapter{C programming: hello world, syntax, and variables}
% \label{Cintro.chap}
% \input tex/me4xxL06

% \chapter{C programming: flow control, functions, scope, structs}
% \label{Cflow.chap}
% \input tex/me4xxL07

% \chapter{C programming: memory}
% \label{Cmemory.chap}
% \input tex/me4xxL08

%% TW: make

% \chapter{C programming: file input-output}
% \label{Cio.chap}
% \input tex/me4xxL09

% \chapter{Debugging: using \texttt{gdb} and \texttt{valgrind}}
% \label{debug.chap}
% \input tex/me4xxL10

% \part{Parallel computing}

% \chapter{High Performance Computing: overview and concepts}
% %\renewcommand\thesection{\thechapter.\arabic{section}}
% \label{HPCIntro.chap}
% \input tex/me4xxL13

% \chapter{Parallel programming with Open Multi-Processing}
% \label{OpenMPIntro.chap}
% \input tex/me4xxL14

% \chapter{OpenMP Part II}
% \label{OpenMPadvanced.chap}
% \input tex/me4xxL15

% \chapter{Parallel programming with the Message Passing Interface}
% \label{MPIIntro.chap}
% \input tex/me4xxL16

% \chapter{Visualizing MPI program execution with the Message Passing Environment (MPE) and Jumpshot}
% \label{MPIViz.chap}
% \input tex/me4xxL17

% \chapter{Using the VT Advanced Research Computing services}
% \label{ARC.chap}
% \input tex/me4xxL18

% \chapter{MPI: collective communications}
% \label{MPICollectives.chap}
% \input tex/me4xxL19

% \chapter{MPI: performance models, analysis, and scalability}
% \label{MPIPerformance.chap}
% \input tex/me4xxL20

% \chapter{Algorithms for fast queries of scattered data}
% \label{quadtree.chap}
% \input tex/me4xxL21

% \chapter{Using parallel R}
% \label{ParallelR.chap}
% \input tex/me4xxL22

% \chapter{Code profiling and optimization}
% \label{performance.chap}
% \input tex/me4xxL23

% \chapter{WIP Why GPUs ?}
% \label{whyGPUs.chap}
% \input tex/me4xxL24

% \chapter{CUDA introduction: hello world}
% \label{CUDAhelloWorld.chap}
% \input tex/me4xxL25

% \chapter{CUDA part 2: GPU sum reduction}
% \label{CUDAreductions.chap}
% \input tex/me4xxL26

% \chapter{Portable GPU computing: AMD HIP and OpenCL}
% \label{advancedGPU.chap}
% \input tex/me4xxL27

% \part{High performance computing with Python}

% \chapter{Python: Introduction}
% \label{PythonIntro.chap}
% \input tex/me4xxL11

% \chapter{Python: Visualizing K-means}
% \label{PythonViz.chap}
% \input tex/me4xxL12

% \chapter{Python: Improving Python performance with Cython }
% \label{pythonCython.chap}

% \chapter{Python: Heterogeneous computing with PyCUDA \& PyOpenCL }
% \label{pythonGPU.chap}
% \input tex/me4xxL28

% \part{Assignments}
% \chapter{Assignments from Fall 2019}
% \label{assignments2019.chap}
% \input tex/me4xxAssignments

% \part{Algorithms}

% \chapter{Fall 2019: exploring finite precision by recurrence}
% \label{finitePrecision.chap}
% \input tex/me4xxFinitePrecision


% \chapter{Fall 2019: clustering data with the k-means algorithm}
% \label{Kmeans.chap}
% \input tex/me4xxKmeans

% \part{Appendix}

% \chapter{CMDA 3634 style guide}
% \input tex/me4xxStyleGuide

\end{document}
